\documentclass[a4paper,12pt]{article}
\usepackage[utf8]{inputenc}
\usepackage{amsmath}
\usepackage{amssymb}
\usepackage{graphicx}
\usepackage{hyperref}
\usepackage{geometry}
\geometry{margin=1in}
\usepackage{enumitem}

\title{Rapport sur le Contrôle et la Simulation du Robot UR10}
\author{Curtis Martelet}
\date{29/11/2024}

\begin{document}

\maketitle
\tableofcontents
\newpage

\section{Introduction}
Ce rapport sert de conclusion au travail réalisé sur la simulation et le contrôle du robot UR10.  
Nous y développons le calcul et l'implémentation d'un système de commande déterminé avec la méthode de Denavit-Hartenberg modifiée.  
Nous décrivons également les étapes de simulation sous \textit{CoppeliaSim} ainsi que les résultats obtenus.  

\section{Modélisation du Robot UR10}
\subsection{Présentation de la méthode DH}
La méthode Denavit-Hartenberg (DH) permet de décrire la géométrie d'un robot manipulateur.  
Chaque articulation est modélisée par une matrice de transformation homogène qui exprime la position et l'orientation d'un repère local par rapport au repère précédent.  

\subsection{Schéma du robot}
\begin{figure}[h!]
    \centering
    \includegraphics[width=0.5\textwidth]{repere.png}
    \caption{Schéma du robot avec repères}
\end{figure}

\subsection{Paramètres DH}
\subsubsection{Simulation}
Dans le cas de la simulation dans \textit{CoppeliaSim}, les paramètres DH sont modifiés pour aligner les axes des repères du simulateur.  

\begin{table}[h!]
    \centering
    \begin{tabular}{|c|c|c|c|c|c|}
    \hline
    Joint & $\sigma_j$ & $\alpha_{j-1}$ (rad) & $a_{j-1}$ (m) & $\theta_j$ (rad) & $d_j$ (m) \\ \hline
    1 & 0 & 0 & 0 & $\theta_1$ & $r_1$ \\ \hline
    2 & 0 & $\pi/2$ & 0 & $\theta_2 - \pi/2$ & $r_2$ \\ \hline
    3 & 0 & 0 & $-a_2$ & $\theta_3$ & 0 \\ \hline
    4 & 0 & 0 & $-a_3$ & $\theta_4$ & 0 \\ \hline
    5 & 0 & $\pi/2$ & 0 & $\theta_5 - \pi/2$ & $r_5$ \\ \hline
    6 & 0 & $-\pi/2$ & 0 & $\theta_6$ & 0 \\ \hline
    \end{tabular}
    \caption{Tableau DH pour la simulation.}
\end{table}

\subsubsection{Robot Réel}
Pour le robot réel, les repères respectent directement la configuration physique du robot, ce qui entraîne une différence au niveau des angles des joints.  

\begin{table}[h!]
    \centering
    \begin{tabular}{|c|c|c|c|c|c|}
    \hline
    Joint & $\sigma_j$ & $\alpha_{j-1}$ (rad) & $a_{j-1}$ (m) & $\theta_j$ (rad) & $d_j$ (m) \\ \hline
    1 & 0 & 0 & 0 & $\theta_1$ & $r_1$ \\ \hline
    2 & 0 & $\pi/2$ & 0 & $\theta_2$ & $r_2$ \\ \hline
    3 & 0 & 0 & $-a_2$ & $\theta_3$ & 0 \\ \hline
    4 & 0 & 0 & $-a_3$ & $\theta_4$ & 0 \\ \hline
    5 & 0 & $\pi/2$ & 0 & $\theta_5$ & $r_5$ \\ \hline
    6 & 0 & $-\pi/2$ & 0 & $\theta_6$ & 0 \\ \hline
    \end{tabular}
    \caption{Tableau DH pour le robot réel.}
\end{table}

Nous avons retiré les décalages de $\pi/2$ sur les articulations 2 et 5.

\section{Formules et Modélisation Mathématique}
Dans cette partie, nous détaillons les étapes et calculs utilisés dans la commande du robot.  
Le système de commande comprend les blocs suivants :
\begin{itemize}
    \item Un générateur de trajectoire.
    \item Un modèle géométrique.
    \item Un calcul de l'erreur.
    \item Un contrôleur pour calculer les vitesses articulaires désirées.
\end{itemize}

\subsection{Générateur de Trajectoire}
\subsubsection{Calcul des positions et vitesses désirées}
Les positions et vitesses désirées sont calculées à l'aide d'un polynôme de cinquième degré :

\[
r(t) = 10 \left( \frac{t}{t_f} \right)^3 - 15 \left( \frac{t}{t_f} \right)^4 + 6 \left( \frac{t}{t_f} \right)^5
\]

\[
r_{\text{point}}(t) = 30 \frac{t^2}{t_f^3} - 60 \frac{t^3}{t_f^4} + 30 \frac{t^4}{t_f^5}
\]

La position et la vitesse désirée sont ensuite interpolées :

\[
x_{\text{désirée}} = x_{\text{init}} + r(t) (x_{\text{final}} - x_{\text{init}})
\]
\[
x_{\text{point désirée}} = r_{\text{point}}(t) (x_{\text{final}} - x_{\text{init}})
\]

\subsubsection{Calcul de l'orientation désirée}
L'orientation désirée est obtenue par interpolation.  

\paragraph{Matrice de rotation relative}
\[
R = R_{\text{init}}^T \cdot R_{\text{final}}
\]

\paragraph{Angle de rotation}
\[
\cos(\theta) = \frac{\text{Tr}(R) - 1}{2}
\]
\[
\sin(\theta) = \frac{\sqrt{(R_{32} - R_{23})^2 + (R_{13} - R_{31})^2 + (R_{21} - R_{12})^2}}{2}
\]
\[
\theta = \tan^{-1}\left(\frac{\sin(\theta)}{\cos(\theta)}\right)
\]

\paragraph{Axe de rotation}
\[
\mathbf{u} = \frac{1}{2\sin(\theta)} 
\begin{bmatrix}
R_{32} - R_{23} \\ R_{13} - R_{31} \\ R_{21} - R_{12}
\end{bmatrix}
\]

\paragraph{Rotation désirée}
\[
R_{\text{désirée}} = R_{\text{init}} \cdot \text{rot}(\mathbf{u}, r \cdot \theta)
\]

\subsection{Modèle Géométrique}
\subsubsection{Les Matrices de Transformation}
La matrice de transformation homogène est donnée par :

\[
T_j = \begin{bmatrix}
\cos\theta_j & -\sin\theta_j \cos\alpha_{j-1} & \sin\theta_j \sin\alpha_{j-1} & a_{j-1} \\
\sin\theta_j & \cos\theta_j \cos\alpha_{j-1} & -\cos\theta_j \sin\alpha_{j-1} & -r_j \sin\alpha_{j-1} \\
0 & \sin\alpha_{j-1} & \cos\alpha_{j-1} & r_j \cos\alpha_{j-1} \\
0 & 0 & 0 & 1
\end{bmatrix}
\]

La transformation totale est :
\[
T_{06} = T_1 \cdot T_2 \cdot T_3 \cdot T_4 \cdot T_5 \cdot T_6
\]

\section{Conclusion}
En conclusion, la simulation du robot UR10 sur \textit{CoppeliaSim} fonctionne, même si une singularité a été rencontrée. Les calculs effectués valident le modèle utilisé. Toutefois, ce programme n'a pas encore été testé sur le robot réel.  
\end{document}
